\documentclass[10pt]{beamer}
\usetheme{metropolis}
% all imports
\input{all_imports}

\AtBeginEnvironment{quote}{\singlespacing}

% new commands
\input{all_new_commands}

% definitions
\input{definitions/colors}
\input{definitions/styles}

\input{header}

\begin{document}
\nocite{DeepLearningbook}


\maketitle

\section{Introduction}
% definicao de IA em uma linha com exemplos 
% mudança de paradigma em IA. Track de IA do Bcc eh enviesado para logica http://bcc.ime.usp.br/principal/ia
% https://uspdigital.usp.br/jupiterweb/obterDisciplina?sgldis=MAC0425&verdis=3
% ML ser requisito ? Cada vez mais uso de dados e ML em IA, a disciplina de IA poderia ser a ponte entre IA classica e ML
% disciplina  de ML  https://uspdigital.usp.br/jupiterweb/obterDisciplina?sgldis=MAC0460&verdis=2
% aceitar cursos online de ML como requisito? cursos muito praticos FORMA MAO DE OBRA ESCRAVA
% NAO ESQUECER CRASE
% muda professores todo semestre com ementas levemente diferentes
%Ementa da Leliane vs. Ementa do Dennis
%logica pode ser optativa

% Leliane

% Busca Geral, Cega, Informada e Local
% Agentes Lógicos (I e II)
% Lógica de Primeira Ordem
% Cálculo de Situações
% Planejamento Automatizado I
% Planejamento Automatizado II - Planejador Progressivo, Regressivo e POP
% Planejamento Automatizado III - GRAPHPLAN e Heurísticas em Planejamento 
% Incerteza \& Raciocínio Probabilístico
% MDP (I e II)
% Aprendizado por Reforço (I e II)
% Aprendizado de Máquina: Redes Neurais 
% Aprendizado de Máquina: Árvore de Decisão
% Seminário de Deep Learning

% Dennis

% [introdução](aulas/intro.pdf) (cap 1 e 2)
% [busca cega](aulas/buscacega.pdf) (cap 3.1 a 3.4)
% [busca informada](aulas/buscainfo.pdf) (cap 3.5)
% [busca local](aulas/buscalocal.pdf) (cap 4)
% [busca competitiva](aulas/minimax.pdf) (cap 5)
% [agentes lógicos e lógica proposicional](aulas/logprop.pdf) (cap 7)
% [lógica de primeira ordem](aulas/lpo.pdf) (cap 8)
% [inferência lógica](aulas/inf.pdf) (cap 7 a 9)
% [incerteza](aulas/uncertainty.pdf) (cap 13)
% [raciocínio probabilístico I](aulas/independence.pdf) (cap 14)
% [raciocínio probabilístico II](aulas/bayesnets.pdf) (cap 14)
% [raciocínio probalístico temporal](aulas/dbn.pdf) (cap 15)
% [tomada de decisão simples](aulas/decision.pdf) (cap 16)
% [tomada de decisão sequencial](aulas/mdp.pdf) (cap 17)  
% [MDP: Algoritmos](aulas/mdp.pdf) (cap 17)  
% [POMDPs](aulas/pomdp.pdf) (cap 17)
% [aprendizagem por reforço](aulas/rl.pdf) (cap 21)
% [aprendizagem por reforço aproximada e regressão](aulas/rl2.pdf) (cap 18 e 21)
% [classificação](aulas/learning.pdf) (cap 18)
% 
%
%
%
%RL une os paradigmas classico e moderno de IA
%apresentar EP do pacman e um novo EP
%
% slide resumindo toda a nossa disrupcao e alteracoes na disciplina
%

\begin{frame}[fragile]{Basic idea}
\begin{itemize}
\item gen eh legal
\item limite eh dificil
\end{itemize}
\end{frame}

\begin{frame}[allowframebreaks]{References}

  \bibliography{my_references}
  \bibliographystyle{abbrv}

\end{frame}

\end{document}
